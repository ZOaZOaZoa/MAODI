\usepackage[russian]{babel}
\usepackage[utf8x]{inputenc}
\newtheorem{theorem}{\hspace*{1.25cm}Теорема}
%Соответствующий математический шрифт для Times new roman
%\usepackage{newtxmath}

%Шрифты
\usepackage{fontspec} 
\defaultfontfeatures{Ligatures={TeX},Renderer=Basic} 
\setmainfont{NewCM10-Book}
\setsansfont{NewCMSans10-Book}
\setmonofont{NewCMMono10-Book}

%\setdefaultlanguage{russian}

%Геометрия
\usepackage{geometry}
\geometry{top=20mm}
\geometry{bottom=15mm}
\geometry{left=20mm}
\geometry{right=10mm}
\usepackage{setspace}
%Нормальные дроби через запятую
\usepackage{ncccomma}

\newcommand{\changefont}{%
	\fontsize{12}{11}\selectfont
}

%Заголовки
\usepackage{fancyhdr}
\pagestyle{fancy}
\fancyhf{}
%\renewcommand{\sectionmark}[1]{\markright{#1}}
\fancyhead[R]{\changefont \slshape \leftmark}
%\fancyhead[L]{\changefont \slshape \rightmark}
%\newcommand{\ssubsection}[1]{\subsection*{#1}
%	\addcontentsline{toc}{subsection}{#1}
%	\markright{#1}{}}
\cfoot{\thepage}

%\полуторный интервал
\setstretch{1.15}
\setlength{\parindent}{1.25cm}

\usepackage{amsmath, amsfonts, mathtools,amssymb}
\usepackage{bm}
\usepackage{physics}
\usepackage{indentfirst}
\usepackage{xcolor}
\usepackage{alltt}
\usepackage{graphicx}
\usepackage{wrapfig}
\usepackage{pgfplots}
\usepackage{multirow}
\usepackage{standalone}  

%Настройка ссылок
\usepackage{hyperref}
\usepackage{listings}
\usepackage{minted}
\lstdefinestyle{python}{
	language=Python,
	breaklines=true,
	frame=single,
	numbers=left,
	keywordstyle=\bfseries\color{green!40!black},
	frame=lines,
	basicstyle=\footnotesize\rmfamily
}
%\usepackage{upgreek}
%\renewcomand{\beta}{\upbeta}
\hypersetup{
	colorlinks,
	citecolor=black,
	filecolor=black,
	linkcolor=black,
	urlcolor=black
}
\usepackage{caption}
\DeclareCaptionLabelSeparator{dot}{. }
\captionsetup{justification=centering,labelsep=dot}
\usepackage{titlesec}

%Формат заголовков
\titleformat{\section}{\bfseries\filcenter\uppercase}{\thesection}{1em}{}
\titleformat{\subsection}{\bfseries}{\thesubsection}{1em}{}
\titleformat{\subsubsection}{\bfseries\filcenter\normalsize}{\thesubsubsection}{1em}{}

\usepackage{chngcntr}

%Включить в нумерацию картинок раздел
\counterwithin{figure}{section}
\renewcommand{\sp}[1]{\mathrm{sp}\,#1}