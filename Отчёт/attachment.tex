\appendix
\titleformat{\section}{\normalfont\large\bfseries}{\centering Приложение \thesection. }{0pt}{\large\centering}
\renewcommand{\thesection}{\Asbuk{section}}
\section{Листинги}

{
	\captionof{lstlisting}{Генерация временных рядов $y_1(t)$, $y_3(t)$}
	\label{generation}
	\begin{minted}[frame=lines,fontsize=\footnotesize,breaklines=true,numbers=left]{python}
import numpy as np
import matplotlib.pyplot as plt
from matplotlib.patches import Ellipse
import matplotlib.transforms as transforms
import scipy.stats as stats
from scipy import signal
from collections import namedtuple

N = 500
T = 1 # seconds
delta = T / N
a0 = 0.4
a1 = 0.015
a2 = 0.63

np.random.seed(42)
e1 = np.random.standard_normal(N + 1)
e2 = np.random.standard_normal(N + 1)

k = np.array(range(N + 1))
t = k * delta
y1 = a0 * np.sin(a1 * 2 * np.pi * t) + a2 * e1

y3 = np.exp(0.1 * t) + 2 * e2
	\end{minted}
}

{
	\captionof{lstlisting}{Функция построения эллипса рассеяния}
	\vspace{-1.5em}
	\label{conf_ellipse}
	\begin{minted}[frame=lines,fontsize=\footnotesize,breaklines=true,numbers=left]{python}
def confidence_ellipse(x, y, ax, p=0.99, facecolor='none', **kwargs):
	if x.size != y.size:
		raise ValueError("x and y must be the same size")
	
	cov = np.cov(x, y)
	pearson = cov[0, 1]/np.sqrt(cov[0, 0] * cov[1, 1])
	# Используется частный случай получения собственных значений
	ell_radius_x = np.sqrt(1 + pearson)
	ell_radius_y = np.sqrt(1 - pearson)
	ellipse = Ellipse((0, 0), width=ell_radius_x * 2, height=ell_radius_y * 2,
	facecolor=facecolor, **kwargs)
	
	# Масштабирование в соответствии с данным pvalue в предположении (x, y) ~ N 
	n_std_for_quantile = stats.norm.ppf((1 + p) / 2)
	
	scale_x = np.sqrt(cov[0, 0]) * n_std_for_quantile
	mean_x = np.mean(x)
	
	scale_y = np.sqrt(cov[1, 1]) * n_std_for_quantile
	mean_y = np.mean(y)
	
	# A, такой что xAx^T = 1 описывает эллипс рассеяния до масштабирования и поворота
	A = np.array([ [1/((ell_radius_x) ** 2), 0], [0, 1/((ell_radius_y) ** 2)]])
	scale = np.array([[1/(scale_x ** 2), 0], [0, 1/ (scale_y ** 2)]])
	
	angle = np.pi / 4
	rotation = np.array([ [np.cos(angle), -np.sin(angle)], [np.sin(angle), np.cos(angle)] ])
	
	A = rotation @ scale @ A @ rotation.T
	mu = np.array([mean_x, mean_y])
	
	transf = transforms.Affine2D() \
		.rotate_deg(45) \
		.scale(scale_x, scale_y) \
		.translate(mean_x, mean_y)
	
	ellipse.set_transform(transf + ax.transData)
	return ax.add_patch(ellipse), A, mu
	\end{minted}
}

{
	\captionof{lstlisting}{Построение эллипса рассеяния}
	\vspace{-1.5em}
	\label{plot_ellipse}
	\begin{minted}[frame=lines,fontsize=\footnotesize,breaklines=true,numbers=left]{python}
def replace_with_mean(arr: np.array, indices):
	result = np.copy(arr)
	for i in indices:
		if i == 0 or (i == result.shape[0] - 1):
			continue
	
		result[i] = 0.5 * (result [i - 1] + result[i + 1])
	return result
	
def ellipse(y, title: str):
	differences = y[1:] - y[:-1]
	forward_diff = differences[1:]
	backward_diff = differences[:-1]
	
	fig, ax = plt.subplots()
	
	_, A, mu = confidence_ellipse(forward_diff, backward_diff, ax, edgecolor='firebrick')
	
	ell_x = np.c_[forward_diff, backward_diff]
	diagonal_values = np.diag((ell_x - mu) @ A @ (ell_x - mu).T)  # вектор длины n
	signes = np.sign(forward_diff * backward_diff)
	anomalies = ell_x[(diagonal_values > 1) & (signes < 1)]
	anomalies_indices = np.where((diagonal_values > 1) & (signes < 1))[0]
	ell_x_clean = replace_with_mean(ell_x, anomalies_indices)
	
	# Добавление на график ell_x_clean, anomalies

ellipse(y1, '$y_1$')
ellipse(y3, '$y_3$')
	\end{minted}
}

{
	\captionof{lstlisting}{Оценка стационарности рядов}
	\vspace{-1.5em}
	\label{stationary_prog}
	\begin{minted}[frame=lines,fontsize=\footnotesize,breaklines=true,numbers=left]{python}
def estimate_m(arr: np.array) -> float:
	N = arr.shape[0]
	return (arr @ np.ones(N)) / N

def estimate_m_var(arr: np.array):
	N = arr.shape[0]
	m = estimate_m(arr)
	return m, ((arr - m).T @ (arr - m)) / (N - 1) 

def series_test(arr: np.array):
	N = arr.shape[0]
	arr = arr[1:] - arr[:-1]

	v, t_list = 1, []
	counter, elem_sign = 1, np.sign(arr[0])
	for x in np.nditer(arr[1:]):
		if elem_sign * x >= 0:
			counter += 1
		else:
			v += 1
			t_list.append(counter)
			counter, elem_sign = 1, np.sign(x)
	t_list.append(counter)
	t = np.max(t_list)
	if N <= 6:
		tmax = 5
	elif N <= 154 and N > 6:
		tmax = 6
	else:
		tmax = 7

	no_trend = (v > ((2 * (N - 1)) / 3 - 1.96 * np.sqrt((16 * N - 29) / 90))) and t < tmax
	return no_trend, v, t

def trend_test(y, title):
	mus = []
	vars = []
	for i in range(10):
	arr = y[50 * i: 50 * (i + 1)]
	mu, var = estimate_m_var(arr)
	mus.append(mu)
	vars.append(var)


	no_trend, _, _ = series_test(np.array(mus))    
	print(f'По результату критерий серий "Тренд в m({title.replace("$", "")}) отсутствует - {no_trend}"')
	no_trend, _, _ = series_test(np.array(vars))    
	print(f'По результату критерий серий "Тренд в var({title.replace("$", "")}) отсутствует - {no_trend}"')
	plt.plot(mus, label="$\mu_{" + title + "}$")
	plt.plot(vars, label="$\sigma_{" + title + "}^2$")
	plt.legend()
	plt.title(f'Изменение параметров реализации {title} со временем')
	plt.xlabel('Номер блока')
	plt.show()

def corr_spectrum(y, title):
	plt.acorr(y, maxlags=30, normed=True)
	plt.title(f'Автокорреляционная функция {title}')
	plt.xlabel('$\\tau$')
	plt.show()

	f, Pxx = signal.periodogram(y.flatten(), scaling='density', fs=1/delta)
	plt.plot(f, Pxx)
	plt.xlabel('Hz')
	plt.ylabel('$V^2/Hz$')
	plt.title(f'Спектральная мощность {title}')
	plt.show()

trend_test(y1, 'y_1')
corr_spectrum(y1, '$y_1$')

trend_test(y3, 'y_3')
corr_spectrum(y3, '$y_3$')

def print_characteristics(arr: np.array):
	mu, var = estimate_m_var(arr)
	skew = stats.skew(arr)
	kurtosis = stats.kurtosis(arr)
	kstest = stats.kstest(arr, 'norm', alternative='two-sided')

print(f'Математическое ожидание - {mu:.3f}')
print(f'Дисперсия - {var:.3f}')
print(f'Асимметрия - {skew:.3f}')
print(f'Эксцесс - {kurtosis:.3f}')
print(f'Тест на нормальность Колмогорова-Смирнова - pvalue={kstest.pvalue:.1e}')

print('Характеристики y1')
print_characteristics(y1)

print('\nХарактеристики y3')
print_characteristics(y3)
	\end{minted}
}


{
	\captionof{lstlisting}{Построение корреляционных функций для разных сигналов}
	\vspace{-1.5em}
	\label{corr_plot_prog}
	\begin{minted}[frame=lines,fontsize=\footnotesize,breaklines=true,numbers=left]{python}
def correlation_function(x: np.array, y: np.array, lags_count: int):
	if x.shape[0] != y.shape[0]:
		raise ValueError('x and y must be the same size')
	N = x.shape[0]
	
	if lags_count > N:
		raise ValueError('lags_count should be less than x, y size')
	
	mx = estimate_m(x)
	my = estimate_m(y)
	Rxy = []
	for k in range(lags_count):
		Rxy.append((x[:N-k] - mx).T @ (y[k:] - my) / N)
	
	return Rxy

def plot_corr_function(x: np.array, y: np.array, lags_count: int, title: str):
	is_autocorr = np.all(x == y)
	
	if is_autocorr:
		Rxy = correlation_function(x, x, lags_count)
	else:
		Rxy = correlation_function(x, y, lags_count)
	
	fig, axs = plt.subplots(1, 2, figsize=(12, 5))
	axs[0].set_title('Сигналы')
	axs[0].plot(x[:100], label='x')
	if not is_autocorr:
		axs[0].plot(y[:100], label='y')
	axs[0].legend()
	
	axs[1].plot(Rxy)
	axs[1].set_title(f'{"Автокорреляционная" if is_autocorr else "Корреляционная"} функция {title}')
	axs[1].set_xlabel('$k\\Delta$')
	axs[1].hlines(0.05, 0, lags_count, color='k', linestyles='dashed')
	plt.show()

def signal_with_garmonics(N, delta = 1e-3):
	k = np.array(range(N))
	return 1 * np.ones(N) + 1.3 * np.sin(200 * 2 * np.pi * k * delta + 0.4) + 0.3 * np.sin(50 * 2 * np.pi * k * delta + 0.93)

def signal_linear_correlated(N, alpha=0.95):
	w = np.random.standard_normal(N)

	x = np.zeros(N)
	x[0] = w[0] / np.sqrt(1 - alpha**2)
	for n in range(1, N):
		x[n] = alpha * x[n-1] + np.sqrt(1 - alpha**2) * w[n]
	return x
	
Realization = namedtuple('Realization', 'generate_x generate_y lags_to_plot')
realizations = {
	'постоянная': Realization(np.ones, None, 10),
	'белый шум': Realization(np.random.standard_normal, None, 10),
	'несколько гармоник': Realization(signal_with_garmonics, None, 100),
	'линейно \nкоррелированные отсчёты': Realization(signal_linear_correlated, None, 100),
	'белый шум \nи несколько гармоник': Realization(np.random.standard_normal, signal_with_garmonics, 100),
	'несколько гармоник \nи коррелированные отсчёты': Realization(signal_with_garmonics, signal_linear_correlated, 100),
	'два сигнала\nс коррелированными отсчётами': Realization(signal_linear_correlated, signal_linear_correlated, 100)
}

N = 100_000

for title, realization in realizations.items():
	x = realization.generate_x(N)
	if realization.generate_y is None:
		plot_corr_function(x, x, realization.lags_to_plot, title)
	else:
		y = realization.generate_y(N)
		plot_corr_function(x, y, realization.lags_to_plot, title)
	\end{minted}
}

{
	\captionof{lstlisting}{Построение спектральных плотностей мощности для разных сигналов}
	\vspace{-1.5em}
	\label{csd_plot_prog}
	\begin{minted}[frame=lines,fontsize=\footnotesize,breaklines=true,numbers=left]{python}
def periodogram(x: np.array, delta):
	N = x.shape[0]
	I = []
	
	i = np.array(range(N))
	for k in range(int(N)):
	alpha = x.T @ np.cos( (2 * np.pi * k * i) / N )
	beta = x.T @ np.sin( (2 * np.pi * k * i) / N )
	I.append(delta / N * (alpha ** 2 + beta ** 2))
	
	return I

# spectral density
def esimate_spe(x: np.array, delta, window):
	N = x.shape[0]
	Sxx = []
	I = np.array(periodogram(x, delta))
	
	j = np.arange(-N / 2, N / 2, 1)
	for k in range(int(N / 2)):
	indices = np.abs( np.astype((k - j), np.int16) )
	Sxx.append( I[indices].T @ window(j) )
	
	return Sxx

def rect_window(j, M = 100):
	w = np.where(np.abs(j) <= M, 1/(2*M + 1), 0)
	return w

def bartlett_window(j, M = 100):
	w = np.where(np.abs(j) <= M, (1 - np.abs(j) / M) / M, 0)
	return w

def hamming_window(j, M = 100):
	w = np.where(np.abs(j) <= M, ( 0.54 + 0.46 * np.cos(np.pi * j / M) ) / (1.08 * M + 0.08), 0)
	return w

def han_window(j, M = 100):
	w = np.where(np.abs(j) <= M, (1 + np.cos(np.pi * j / M)) / (2 * M), 0)
	return w
	
realizations = {
	'постоянная': Realization(np.ones, None, 10),
	'белый шум': Realization(np.random.standard_normal, None, 10),
	'несколько гармоник': Realization(signal_with_garmonics, None, 100),
	'линейно \nкоррелированные отсчёты': Realization(signal_linear_correlated, None, 100),
}

windows = {
	'Прямоугольное окно': rect_window,
	'Окно Бартлетта': bartlett_window,
	'Окно Хэмминга': hamming_window,
	'Окно Хана': han_window,
}

N = 10000
delta = 1e-3
for title, realization in realizations.items():
	x = realization.generate_x(N)
	
	fig, axs = plt.subplots(2, 2, figsize=(12, 8))
	for i, window_name in enumerate(windows):
		Sxx = esimate_spe(x, delta, windows[window_name])
		f = np.arange(len(Sxx)) / (delta * N)
		
		plot_index = int(i / 2), i % 2
		axs[plot_index].set_title(window_name)
		axs[plot_index].plot(f, Sxx)
		
	plt.suptitle(title)
	plt.show()
	\end{minted}
}

{
	\captionof{lstlisting}{Построение спектральных плотностей мощности для разных сигналов}
	\vspace{-1.5em}
	\label{csd_plot_prog2}
	\begin{minted}[frame=lines,fontsize=\footnotesize,breaklines=true,numbers=left]{python}
N = 1500
T = 3 # seconds
delta = T / N
fs = 1 / delta

k = np.array(range(N + 1))
t = k * delta

e = np.random.standard_normal(N + 1)
x = 2 * np.sin(102 * t * 2 * np.pi) + 1.7 * np.sin(102.08 * t * 2 * np.pi) + 2.3 * np.sin(110 * t * 2 * np.pi) + 0.2 * e
y = 1.6 * np.sin(102.8 * t * 2 * np.pi) + 2.1 * np.sin(110 * t * 2 * np.pi) + 2 * np.sin(210 * t * 2 * np.pi) + 0.2 * e
plt.plot(t[:200], x[:200], label='$x$')
plt.plot(t[:200], y[:200], label='$y$')
plt.legend()
plt.show()

# Compute and plot the magnitude of the cross spectral density:
nperseg, noverlap, win = 500, 30, 'hann'


def plot_CSD(csd, csd_name: str):
	fig0, ax0 = plt.subplots(tight_layout=True)
	ax0.set_title(f"{csd_name} ({win.title()}-window, {nperseg=}, {noverlap=})")
	ax0.set(xlabel="Frequency $f$ in Hz", ylabel="CSD Magnitude in V²/Hz")
	ax0.plot(f, np.abs(csd))
	ax0.grid()
	plt.show()

f, Pxx = signal.csd(x, x, fs, win, nperseg, noverlap)
plot_CSD(Pxx, 'Pxx')
f, Pyy = signal.csd(y, y, fs, win, nperseg, noverlap)
plot_CSD(Pyy, 'Pyy')
f, Pxy = signal.csd(x, y, fs, win, nperseg, noverlap)
plot_CSD(Pxy, 'Pxy')
	\end{minted}
}