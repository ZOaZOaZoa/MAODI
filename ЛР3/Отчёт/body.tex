\section{ Выполнение работы }
\subsection{ Формирование выборки векторизованных голосовых команд }

В качестве исходных данных были взяты записи набора голосовых команд: \textbf{<<вперед, назад, налево, направо, стоп, разворот, вверх, вниз, увеличить, уменьшить, сканировать, осмотреться>>}. У каждой команды было 10 отдельных записей. Итого получается выборка из 120 голосовых команд.

В качестве набора информативных признаков голосовой команды были выбраны усреднённые мелчастотные кестральные коэффициенты (mfcc) по набору кадров длиной 10 мс из исходной записи голосовой команды. Примеры визуализаций векторизованных команд приведены на рис. \ref{mfcc_similar}-\ref{mfcc_different}.

\begin{figure}[h!]
	\centering\includegraphics[width=.7\textwidth]{./png/mfcc1.png}
	\caption{Усреднённые mfcc для двух одинаковых команд <<назад>>}
	\label{mfcc_similar}
\end{figure}

\begin{figure}[h!]
	\centering\includegraphics[width=.7\textwidth]{./png/mfcc2.png}
	\caption{Усреднённые mfcc для двух разных команд <<назад>> и <<разворот>>}
	\label{mfcc_different}
\end{figure}

Визуально различия графики для разных команд имеют похожий характер, однако по ним может быть проведена классификация.

\begin{figure}[h!]
	\centering\includegraphics[width=.7\textwidth]{./png/mfcc_vocal.png}
	\caption{Усреднённые mfcc для вокализованного участка и не вокализованного}
	\label{mfcc_voczl}
\end{figure}

Для дальнейшей классификации были выделены тренировочная и тестовая выборки в соотношении 4 к 1.

\subsection{ Классификация с помощью DTW }

Метод динамической трансформации шкалы времени (DTW) позволяет получить путь деформации между двумя голосовыми командами. Он представляет собой минимальный путь. Длина пути деформации, по сути, представляет собой некую меру близости двух голосовых команд.

Для классификации рассчитывается длина пути деформации между классифицируемой голосовой командой и всеми голосовыми командами из тренировочной выборки. Ей присваивается метка голосовой команды ближайшей по данной метрике. Классификатор был реализован в виде класса, представленного на листинге \ref{DTW_classifier}.

По своей сути, реализованный классификатор на основе DTW это метод 1-го ближайшего соседа с использованием длины пути деформации в качестве метрики расстояния.

\subsection{ Классификация с помощью многослойного персептрона }

Классификатор был реализован в виде класса NN, представленного в листинге \ref{NN_classifier}. В своей структуре это персептрон с двумя скрытыми слоями с 32 и 16 нейронами соответственно и функциями активации RELU и LeakyRELU соответственно. На выходе используется 12 нейронов с функцией активации softmax, т.к. производится классификация 12 голосовых команд.

Обучение производится с помощью алгоритма оптимизации adam с использованием категориальной кросс-энтропии в качестве минимизируемой метрики. Для обучения из тренировочной выборки дополнительно было выделено 20\% голосовых команд в валидационную выборку. Обучение производилось до момента времени, пока функция потери на валидационной выборке не станет расти в течение 3 эпох подряд.

\subsection{Полученные результаты}

Для сравнения возможностей данных классификаторов в разных условиях был реализован класс Pipeline, описанный в листинге \ref{Pipeline_code}. Он обеспечивал общий интерфейс взаимодействия с обоими классификаторами.

Затем была написана программа, представленная в листинге \ref{Classifiers_tests}, которая обучила ряд классификаторов в различных условиях и сформировала сводную таблицу (листинг \ref{classifiers_table}). По этой таблице были построены графики \ref{dtw_acc}-\ref{nn_time}.

{
	\captionof{lstlisting}{Сводная таблица обученных классификаторов голосовых команд}
	\label{classifiers_table}
	\begin{minted}[frame=lines,fontsize=\scriptsize,breaklines=true]{text}
№  Степень перекрытия  Кол-во инф. признаков  DTW точность  NN точность  DTW время, мс  NN время, мс
1                0.25                      6      0.791667     0.541667       208.1284      164.0788
2                0.25                     12      0.791667     0.916667       554.6691      162.7194
3                0.25                     18      0.875000     0.833333       916.7622      188.6156
4                0.50                      6      0.791667     0.583333       220.4520      188.1650
5                0.50                     12      0.791667     0.833333       599.4209      183.5267
6                0.50                     18      0.833333     0.916667       979.2463      166.5307
7                0.75                      6      0.708333     0.583333       191.3799      164.2450
8                0.75                     12      0.791667     0.875000       520.1233      424.2866
9                0.75                     18      0.875000     0.916667       831.2984      165.6572
	\end{minted}
}

{
\noindent\begin{minipage}{.5\textwidth}
	\centering\includegraphics[width=.95\textwidth]{./png/dtw_acc.png}
	\captionof{figure}{Графики зависимости точности DTW от условий}
	\label{dtw_acc}
\end{minipage}
\begin{minipage}{.5\textwidth}
	\centering\includegraphics[width=.95\textwidth]{./png/dtw_time.png}
	\captionof{figure}{Графики зависимости быстродействия DTW от условий}
	\label{dtw_time}
\end{minipage}

\noindent\begin{minipage}{.5\textwidth}
	\centering\includegraphics[width=.95\textwidth]{./png/nn_acc.png}
	\captionof{figure}{Графики зависимости точности NN от условий}
	\label{nn_acc}
\end{minipage}
\begin{minipage}{.5\textwidth}
	\centering\includegraphics[width=.95\textwidth]{./png/nn_time.png}
	\captionof{figure}{Графики зависимости быстродействия NN от условий}
	\label{nn_time}
\end{minipage}
}

Увеличение степени перекрытия или количества признаков сказывается на быстродействии, но может оказать хорошее воздействие на точности. Для нейросетевого классификатора эти графики не совсем однозначные, поскольку итоговая точность классификатора сильно зависит от исходных весов нейросети, и на практике варьировалась в диапазоне 10\%.

Однако стоит отметить, что наиболее критичным было малое количество признаков для итоговой точности.

\section{Выводы}

В данной лабораторной работе были реализованы два различных классификатора голосовых команд. Один основан на методе динамической трансформации шкалы времени (DTW), и был реализован на основе метода вида 1-го ближайшего соседа. Второй классификатор был реализован в виде многослойного персептрона.

Были обучено 18 классификаторов для сравнения их показателей в различных условиях. 