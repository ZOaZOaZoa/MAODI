\section{ Выполнение работы }
\subsection{ Предобработка изображения }

Предобработка представляет собой сглаживание, бинаризацию и выделение контуров на изображении. Реализованные функции предобработки изображения представлены в листинге \ref{predobr}. Для получения эталонных контуров использовалось изображение, представленное на рис. \ref{sample_fig}. Результат предобработки до выделения контуров представлен на рис. \ref{predobr_sample_fig}. Выделенные контуры представлены на рис. \ref{sample_contourized}.

\begin{figure}[h]
	\centering\includegraphics[width=.8\textwidth]{./png/sample.png}
	\caption{Изображение с эталонными символами}
	\label{sample_fig}
\end{figure}

\begin{figure}[h]
	\centering\includegraphics[width=.8\textwidth]{./png/sample_predobr.png}
	\caption{Предобработанное изображение с эталонными символами}
	\label{predobr_sample_fig}
\end{figure}

\begin{figure}[h]
	\centering\includegraphics[width=.8\textwidth]{./png/sample_contourized.png}
	\caption{Выделенные контуры на эталонном изображении}
	\label{sample_contourized}
\end{figure}

Из набора выделенных эталонных контуров была собрана библиотека эталонных контуров для каждого возможного символа. Именно они будут использоваться для сравнения с выделенными контурами на реальных изображениях.

При получении контуров реального изображения производится исключение контуров со слишком малой площадью, большой длиной. А также контуры упорядочиваются по координате $x$ на изображении.

\subsection{Реализация контурного анализа}

Для применения аппарата теории контурного анализа был реализован класс Contour, описанный в листинге \ref{Contour_class}. В нём реализованы основные операции над контурами. Приведём основную информацию. 

Контур описывается в виде вектора комплексных чисел:

\begin{equation}
	\Gamma = \begin{bmatrix}
		\gamma_1 & \gamma_2 & \dots & \gamma_n
	\end{bmatrix} ^\intercal,\;\gamma_i\in\mathbb{C}
\end{equation}
 
Здесь $\gamma_i$ -- элементарный вектор (ЭВ), описывающий переход от $i-1$ точки контура, к $i$ точке (для контура с $n+1$ точками, начинающимися с 0-го номера).
$\Gamma$ -- вектор-контур, описывающий целый контур.

Для вектор-контуров определено скалярное произведение, как сумма скалярных произведений их элементов:
\begin{equation}
	(\Gamma,\,N) = \sum_{n=0}^{k-1} (\gamma_n,\,\nu_n)
\end{equation}

За длину вектора принимается сумма длин элементарных векторов:
\begin{equation}
	|\Gamma| = \norm{\Gamma}_2 = \sqrt{(\Gamma, \overline{\Gamma})}
\end{equation}
Здесь $\overline{\Gamma}$ -- комплексно-сопряжённое значению $\Gamma$.

Для сравнения подобия двух контуров используется взимная корреляционная функция (ВКФ):
\begin{equation}
	\tau(m) = (\Gamma,\,N^{(m)})
\end{equation}
Где $N^{(m)}$ -- вектор, полученный из $N$ циклическим сдвигом его элементов налево на $m$ позиций.

В качестве меры схожести двух контуров использовалась следующая метрика:
\begin{equation}
	\tau_{\max} = \frac{1}{|\Gamma|\cdot |N|}\max{\tau(m)}
	\label{tau_max}
\end{equation}

Для выравнивания длин вектор-контуров, требуемая определением скалярного произведения, была реализована функция эквализации вектор-контура.

Она позволяет получить схожий вектор-контур по своим характеристикам, но с заданным количеством элементарных векторов в нём.

\subsection{Классификация символов автомобильных номеров}

Для классификации символов на автомобильных номерах был реализован алгоритм, представленный в листинге \ref{Character_classification}. Приведём пример работы алгоритма на следующем изображении (рис. \ref{classification_sample}). 

Алгоритм определяет схожесть каждого контура с каждым эталонным в соответствии с (\ref{tau_max}). Если наиболее схожий контур имеет значение $\tau_{\max}>0,5$, то ему назначается метка, такая же как у эталонного контура.

Сами контуры упорядочены по $x$ координате, первые 6 классифицированных контуров составляется в предсказание номера. В данном примере номер составлен довольно похоже, но с ошибкой. Такой подход к классификации контуров может путать букву "в" с "8".

Также можно обратить внимание, что алгоритм с большой степенью уверенности классифицировал "6" из кода региона как "9". Это связано с тем, что контурный анализ позволяет определить похожие контуры инвариантно к повороту.

\begin{figure}[h]
	\centering\includegraphics[width=.8\textwidth]{./png/classification_sample.png}
	\caption{Пример классификации на реальном номере}
	\label{classification_sample}
\end{figure}


{
	\captionof{lstlisting}{Вывод алгоритма классификации контуров}
	\label{classification_log}
	\begin{minted}[frame=lines,fontsize=\footnotesize,breaklines=true,numbers=left]{text}
Контур   1 наиболее похож на   8 со схожестью  0.72
Контур   2 наиболее похож на   1 со схожестью  0.71
Контур   3 наиболее похож на   7 со схожестью  0.68
Контур   4 наиболее похож на   0 со схожестью  0.73
Контур   6 наиболее похож на   M со схожестью  0.61
Контур   7 наиболее похож на   M со схожестью  0.61
Контур   8 наиболее похож на   1 со схожестью  0.83
Контур   9 наиболее похож на   9 со схожестью  0.84
Контур  10 наиболее похож на   2 со схожестью  0.54
Контур  11 наиболее похож на   1 со схожестью  0.83
Контур  13 наиболее похож на   8 со схожестью  0.81

Получили номер 8170MM
Чтение и предобратотка изображения: 196мс
Отсев лишних контуров и сортировка по координатам: 11мс
Отрисовка контуров на изображении: 194мс
Классификация контуров: 1699мс
	\end{minted}
}

\section{Выводы}

В этой лабораторной работе были рассмотрены и реализованы функции для применения инструментов контурного анализа изображений. На основе этого метода была реализована классификация символов на автомобильных номерах.

Выделенные контуры получилось классифицировать в соответствии с возможными символами, однако не для всех символов такой подход хорошо работает. Например, цифры 6 и 9 путаются алгоритмом из-за инвариантности к повороту.